\chapter{Linear Orderings}

\begin{exercise}
	Give an example of a relation that is neither strict nor reflexive.
\end{exercise}

\begin{solution}
	The equality of IEEE floating-point numbers is neither strict nor reflexive. To show this, we first
	demonstrate that the relation is not strict; if we compare a well defined value, such as 0, then
	the strict property will not hold since \verb|(!(0 == 0)) == false|. Second, we need to show that
	the relation is irreflexive, we can do this if we compare \verb|NAN| for equality with itself, then
	\verb|(NAN == NAN) == false|.

	\lstinputlisting[language=c++, firstline=7, lastline=9]{code/ex_4_1.cpp}
\end{solution}

\begin{exercise}
	Give an example of a symmetric relation that is not transitive.
\end{exercise}

\begin{exercise}
	Give an example of a symmetric relation that is not reflexive.
\end{exercise}

\begin{lemma}
	If $r$ is an equivalence relation, $a = b \Rightarrow r(a, b)$.
\end{lemma}

\begin{lemma}
	$\func{key\_function}(f, r) \Rightarrow \func{equivalence}(r)$
\end{lemma}

\begin{lemma}
	The symmetric complement of a weak ordering is an equivalence relation.
\end{lemma}

\begin{lemma}
	A total ordering is a weak ordering.
\end{lemma}

\begin{lemma}
	A weak ordering is asymmetric.
\end{lemma}

\begin{lemma}
	A weak ordering is strict.
\end{lemma}

\begin{exercise}
	Implement \verb|select_2_4|.
\end{exercise}

\begin{lemma}
	\verb|select_2_5| performs six comparisons.
\end{lemma}

\begin{exercise}
	Find an algorithm for median of 5 that does slightly fewer comparisons
	on average.
\end{exercise}

\begin{exercise}
	Prove the stability of every order-selection procedure in this
	section.
\end{exercise}

\begin{exercise}
	Verify the correctness and stability of every order-selection
	procedure in this section by exhaustive testing.
\end{exercise}

\begin{project}
	Design a set of necessary and sufficient conditions preserving
	stability under composition of order-selection procedures.
\end{project}

\begin{project}
	Create a library of minimum-comparison procedures for stable
	sorting and merging. Minimize not only the number of comparisons
	but also the number of data movements.
\end{project}

\begin{exercise}
	Rewrite the algorithms in this chapter using three-valued
	comparison.
\end{exercise}
