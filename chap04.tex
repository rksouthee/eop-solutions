\chapter{Linear Orderings}

\begin{exercise}
	Give an example of a relation that is neither strict nor reflexive.
\end{exercise}

\begin{solution}
	The equality of IEEE floating-point numbers is neither strict nor reflexive. To show this, we first
	demonstrate that the relation is not strict; if we compare a well defined value, such as 0, then
	the strict property will not hold since \verb|(!(0 == 0)) == false|. Second, we need to show that
	the relation is irreflexive, we can do this if we compare \verb|NAN| for equality with itself, then
	\verb|(NAN == NAN) == false|.

	\lstinputlisting[language=c++, firstline=7, lastline=9]{code/ex_4_1.cpp}
\end{solution}

\begin{exercise}
	Give an example of a symmetric relation that is not transitive.
\end{exercise}

\begin{exercise}
	Give an example of a symmetric relation that is not reflexive.
\end{exercise}

\begin{lemma}
	If $r$ is an equivalence relation, $a = b \Rightarrow r(a, b)$.
\end{lemma}

\begin{lemma}
	$\func{key\_function}(f, r) \Rightarrow \func{equivalence}(r)$
\end{lemma}

\begin{lemma}
	The symmetric complement of a weak ordering is an equivalence relation.
\end{lemma}
