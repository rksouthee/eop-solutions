
\chapter{Transformations and Their Orbits}

\begin{lemma}
	$\mathtt{euclidean\_norm}(x, y, z) = \mathtt{euclidean\_norm}(\mathtt{euclidean\_norm}(x, y), z)$
\end{lemma}

\begin{proof}
	\begin{align*}
		\mathtt{euclidean\_norm}(\mathtt{euclidean\_norm}(x, y), z)
		&= \sqrt{\left (\sqrt{x^2 + y^2} \right )^2 + z^2}\\
		&= \sqrt{x^2 + y^2 + z^2}\\
		&= \mathtt{euclidean\_norm}(x, y, z)
	\end{align*}
\end{proof}

\begin{exercise}
	Implement a definition-space predicate for addition on 32-bit signed integers.
\end{exercise}

\lstinputlisting[language=c++, firstline=176, lastline=181]{eop-code/eop.h}

\begin{lemma}
	An orbit does not contain both a cyclic and a terminal element.
\end{lemma}

\begin{proof}
	Let $x$ be cyclic under $f$, that is, $x = f^n(x)$ for $n \geq 1$ and let $y$ be
	terminal under $f$. Since $x$ and $y$ belong to the same orbit, and $y$ is a
	terminal element, $y$ must be reachable from $x$. Since $y$ is reachable from $x$
	we have $y = f^m(x)$ for $m \geq 0$. If $m < n$ then $x = f^{n - m}(y)$, but
	$n - m \geq 1$ and since $y$ is terminal $f^{n - m}(y)$ is not defined. If $n = m$
	then $x = y$, then we could substitute $y$ for $x$, that is $x = f^n(y)$, similarly
	since $n \geq 1$, $f(y)$ is not defined. Finally, if $m > n$, then $x = f^{m - n}(y)$
	but $y$ is not in the definition space of $f$.
\end{proof}

\begin{lemma}
	An orbit contains at most one terminal element.
\end{lemma}

\begin{proof}
	Let us assume that an orbit contains more than one terminal element. For simplicity,
	let $x$ and $y$ be distinct elements, terminal under $f$. Since $x$ and $y$ belong
	to the same orbit, they have a common element they're reachable from, let $z$ be
	such an element, then $x = f^n(z)$ and $y = f^m(z)$. Without loss of generality,
	$n < m$, then $y = f^{m - n}(x)$ but since $m - n \geq 1$ and $x$ is terminal, then
	$f^{m - n}(x)$ is not defined.
\end{proof}

\begin{lemma}
	$o = h + c$
\end{lemma}

\begin{proof}
	Since we defined the orbit handle as the complement of the orbit cycle with respect
	to the cycle, then, $h = o - c$, therefore $o = h + c$.
\end{proof}

\begin{lemma}
	The distance from any point in an orbit to a point in a cycle
	of that orbit is always defined.
\end{lemma}

\begin{proof}
	For an element in an orbit, the distance to an element in the cycle is equal to the
	distance to the connection point plus the distance from the connection point to the
	element in the cycle.
\end{proof}

\begin{lemma}
	If $x$ and $y$ are distinct points in a cycle of size $c$,
	
	\[ c = \mathtt{distance}(x, y, f) + \mathtt{distance}(y, x, f) \]
\end{lemma}

\begin{lemma}
	If $x$ and $y$ are points in a cycle of size $c$, the distance from $x$
	to $y$ satisfies

	\[ 0 \leq \mathtt{distance}(x, y, f) < c \]
\end{lemma}

\begin{lemma}
	If the orbits of two elements intersect, they have the same
	cyclic elements.
\end{lemma}

\begin{exercise}
	Design an algorithm that determines, given a transformation and its definition-space predicate,
	whether the orbits of two elements intersect.
\end{exercise}

\lstinputlisting[language=c++, firstline=3, lastline=12]{code/ex_2_2.cpp}

\begin{exercise}
	The precondition of \verb|convergent_point| ensures termination. Implement an algorithm
	\verb|convergent_point_guarded| for use when that precondition is not known to hold, but
	there is an element in common to the orbits of both $x0$ and $x1$.
\end{exercise}

\lstinputlisting[language=c++, firstline=312, lastline=325]{eop-code/eop.h}

\begin{exercise}
	Derive formulas for the count of different operations (\verb|f|, \verb|p|, equality) for
	the algorithms in this chapter.
\end{exercise}

\begin{exercise}
	Use \verb|orbit_structure_nonterminating_orbit| to determine the average handle size and
	cycle size of the pseudorandom number generators on your platform for various seeds.
\end{exercise}

\begin{exercise}
	Rewrite all the algorithms in this chapter in terms of actions.
\end{exercise}

\begin{project}
	Another way to detect a cycle is to repeatedly test a single advancing element for
	equality with a stored element while replacing the stored element at ever-increasing
	intervals. This and other ideas are described in Sedgewick, et al. [1979], Brent [1980],
	and Levy [1982]. Implement other algorithms for orbit analysis, compare their
	performance for different applications, and develop a set of recommendations for
	selecting the appropriate algorithm.
\end{project}
