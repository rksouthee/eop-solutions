\chapter{Coordinate Structures}

\begin{lemma}
	$\func{height\_recursive}(x) \leq \func{weight\_recursive}(x)$
\end{lemma}

\begin{proof}
	In the case that $x$ is empty we have
	\[ \func{height\_recursive}(x) = \func{weight\_recursive}(x) = 0. \]
	For the case when $x$ is non-empty we just need to show that
	$\func{max}(l, r) \leq l + r$. If $l = \func{max}(l, r)$ then we have
	\[ l \leq l + r, \]
	otherwise, if $\func{max}(l, r) = r$ we have
	\[ r \leq l + r. \]
\end{proof}

\begin{lemma}
	If $x$ and $y$ are bidirectional bifurcate coordinates,
	\begin{align*}
		\func{left\_successor}(x) = \func{left\_successor}(y) &\Rightarrow x = y\\
		\func{left\_successor}(x) = \func{right\_successor}(y) &\Rightarrow x = y\\
		\func{right\_successor}(x) = \func{right\_successor}(y) &\Rightarrow x = y
	\end{align*}
\end{lemma}

\begin{proof}
	Using the regularity of \verb|predecessor| we have
	\begin{align*}
		\func{predecessor}(\func{left\_successor}(x)) &= \func{predecessor}(\func{right\_successor}(x)) = x\\
		\func{predecessor}(\func{left\_successor}(y)) &= \func{predecessor}(\func{right\_successor}(y)) = y
	\end{align*}
\end{proof}

\begin{exercise}
	Would the existence of a coordinate $x$ such that
	\[ \func{is\_left\_successor}(x) \wedge \func{is\_right\_successor}(x) \]
	contradict the axioms of bidirectional bifurcate coordinates?
\end{exercise}

\begin{solution}
	The existence of $x$ does not contradict the axioms of bidirectional bifurcate coordinates.
\end{solution}

\begin{lemma}
	If \verb|reachable| returns true, $v = \func{pre}$ right before the return.
\end{lemma}

\begin{exercise}
	Change \verb|weight| to count \verb|in| or \verb|post| visits instead of \verb|pre|.
\end{exercise}

\lstinputlisting[language=c++, firstline=2640, lastline=2672]{eop-code/eop.h}

\begin{exercise}
	Use \verb|traverse_step| and the procedures of Chapter 2 to determine whether the
	descendants of a bidirectional bifurcate coordinate form a DAG.
\end{exercise}

\lstinputlisting[language=c++, firstline=2715, lastline=2747]{eop-code/eop.h}

\begin{project}
	Implement versions of algorithms in Chapter 6 for bidirectional bifurcate coordinates.
\end{project}

\begin{project}
	Design an adaptor type that, given a bidirectional bifurcate coordinate type, produces an iterator type
	that accesses coordinates in a traversal order (\verb|pre|, \verb|in|, or \verb|post|) specified when an
	iterator is constructed.
\end{project}

\begin{exercise}
	Explain why, in \verb|lexicographical_compare|, the third and fourth \verb|if| statements could be
	interchanged, but the first and second cannot.
\end{exercise}

\begin{solution}
	We can interchange the third and fourth statements because of the weak trichotomy law, both statements
	cannot be true and so the order we check them does not affect correctness. The first and second lines
	cannot be interchanged because they are independent, and if we have exhausted the second range $[f1, l1)$
	then we know that the first range $[f0, l0)$ cannot come before the second range in a lexicographical
	ordering. If we have ranges that are equivalent and we interchange the first and second statements we would
	then consider the first range to come before the second range.
\end{solution}

\begin{exercise}
	Explain why we did not implement \verb|lexicographical_compare| by using \verb|find_mismatch|.
\end{exercise}

\begin{solution}
	Using \verb|find_mismatch| only finds if and where the ranges differ, and so would require more
	iterators comparisons and comparisons for \verb|lexicographical_compare| to determine the ordering.
\end{solution}

\begin{exercise}
	Implement \verb|bifurcate_compare_nonempty| for readable bifurcate coordinates.
\end{exercise}

\lstinputlisting[language=c++, firstline=3017, lastline=3044]{eop-code/eop.h}

\begin{project}
	Design a coordinate structure for a family of data structures, and extend isomorphism, equivalence,
	and ordering to this coordinate structure.
\end{project}
