
\chapter{Foundations}

\begin{lemma}
	If a value type is uniquely represented, equality implies representational equality.
\end{lemma}

\begin{proof}
	Let $T$ be a value type that is uniquely represented. We are going to assume equality,
	then we shall prove representational equality. If two values represent the same abstract entity,
	they must have the same value since $T$ is uniquely represented. If the values are the same, the datums
	are identical sequences of 0s and 1s, that is, they are representationally equal.
\end{proof}

\begin{lemma}
	If a value type is not ambiguous, representational equality implies equality.
\end{lemma}

\begin{proof}
	Let $T$ be a value type that is not ambiguous. We are going to assume that two values of type $T$ are
	representationally equal, then we will prove they represent the same abstract entity. Since $T$ is
	unambiguous and the values have the same datums, they have the same interpretation, that is, they
	represent the same abstract entity.
\end{proof}

\begin{exercise}
	Extend the notion of regularity to input/output objects of a procedure, that is, to objects that are
	modified as well as read.
\end{exercise}
