
\chapter{Associative Operations}


\begin{lemma}
	$a^na^m = a^ma^n = a^{n+m}$ (powers of the same element commute)
\end{lemma}

\begin{proof}
	Because all the groupings are equivalent we can move around the parenthesis and eventually remove
	the parentheses leaving us with $n + m$ elements.

	\begin{align*}
		a^na^m
		&= (\underbrace{aa \ldots aa}_{n \text{ times}})(\underbrace{aa \dots aa}_{m \text{ times}})\\
		&= (\underbrace{aa \ldots aa}_{m \text{ times}})(\underbrace{aa \dots aa}_{n \text{ times}})\\
		&= a^{m + n}\\
		&= a^{n + m}
	\end{align*}
\end{proof}

\begin{lemma}
	$(a^n)^m = a^{nm}$
\end{lemma}

\begin{lemma}
	The binary operation of composition is associative.
\end{lemma}

\begin{lemma}
	\verb|collision_point_nonterminating_orbit| can be used in the proof.
\end{lemma}
