
\chapter{Transformations and Their Orbits}

\begin{lemma}
	$\mathtt{euclidean\_norm}(x, y, z) = \mathtt{euclidean\_norm}(\mathtt{euclidean\_norm}(x, y), z)$
\end{lemma}

\begin{proof}
	\begin{align*}
		\mathtt{euclidean\_norm}(\mathtt{euclidean\_norm}(x, y), z)
		&= \sqrt{\left (\sqrt{x^2 + y^2} \right )^2 + z^2}\\
		&= \sqrt{x^2 + y^2 + z^2}\\
		&= \mathtt{euclidean\_norm}(x, y, z)
	\end{align*}
\end{proof}

\begin{exercise}
	Implement a definition-space predicate for addition on 32-bit signed integers.
\end{exercise}

\lstinputlisting[language=c++]{code/ex_2_1.cpp}

\begin{lemma}
	An orbit does not contain both a cyclic and a terminal element.
\end{lemma}

\begin{lemma}
	An orbit contains at most one terminal element.
\end{lemma}

\begin{lemma}
	$o = h + c$
\end{lemma}

\begin{lemma}
	The distance from any point in an orbit to a point in a cycle
	of that orbit is always defined.
\end{lemma}

\begin{lemma}
	If $x$ and $y$ are distinct points in a cycle of size $c$,
	
	\[ c = \mathtt{distance}(x, y, f) + \mathtt{distance}(y, x, f) \]
\end{lemma}

\begin{lemma}
	If $x$ and $y$ are points in a cycle of size $c$, the distance from $x$
	to $y$ satisfies

	\[ 0 \leq \mathtt{distance}(x, y, f) < c \]
\end{lemma}

\begin{lemma}
	If the orbits of two elements intersect, they have the same
	cyclic elements.
\end{lemma}

\begin{exercise}
	Design an algorithm that determines, given a transformation and its definition-space predicate,
	whether the orbits of two elements intersect.
\end{exercise}
