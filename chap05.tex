\chapter{Ordered Algebraic Structures}

\begin{lemma}
	An identity element is unique:
	\[ \func{identity\_element}(e, op) \wedge \func{identity\_element}(e', op) \Rightarrow e = e' \]
\end{lemma}

\begin{proof}
	\begin{align*}
		e &= op(e, e')\\
		  &= e'
	\end{align*}
\end{proof}

\begin{lemma}
	$n^3$ is the multiplicative inverse modulo 5 of a positive integer $n \neq 0$.
\end{lemma}

\begin{proof}
	Using Fermat's Little Theorem which states that if $p$ is prime, then for any integer $n$
	\[ n^p \equiv n\ (\textrm{mod}\ p). \]
	In our case $p = 5$, and since $n \neq 0$ we have
	\begin{align*}
		n^5 &\equiv n\ (\textrm{mod}\ 5)\\
		n^4 &\equiv 1\ (\textrm{mod}\ 5)\\
		nn^3 &\equiv 1\ (\textrm{mod}\ 5)
	\end{align*}
\end{proof}

\begin{lemma}
	In an additive group, $-0 = 0$.
\end{lemma}

\begin{proof}
	\begin{align*}
		0 &= 0 + (-0) && \text{inverse operation}\\
		  &= (-0) + 0 && \text{commutativity}\\
		  &= -0       && \text{additive identitiy}
	\end{align*}
\end{proof}

\begin{lemma}
	Every additive group is a module over integers with an appropriately
	define scalar multiplication.
\end{lemma}
