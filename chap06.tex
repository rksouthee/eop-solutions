\chapter{Iterators}

\begin{lemma}
	$0 \leq j \leq i \wedge \func{weak\_range}(f, i) \Rightarrow \func{weak\_range}(f, j)$
\end{lemma}

\begin{proof}
	$(0 \leq k \leq j) \Rightarrow \func{successor}^k(f) \text{ is defined.}$
\end{proof}

\begin{lemma}
	$(f + n) + m = f + (n + m)$
\end{lemma}

\begin{proof}
	\begin{align*}
		f + 0 &= f\\
		f + \func{successor}(n) &= \func{successor}(\func{successor}^n(f))
	\end{align*}
\end{proof}

\begin{lemma}
	\verb|successor| is defined for every iterator in a half-open range and for every iterator
	except the last in a closed range.
\end{lemma}

\begin{lemma}
	If $i \in [f, l)$, both $[f, i)$ and $[i, l)$ are bounded ranges.
\end{lemma}

\begin{proof}
	The range $[f, l)$ is partitioned into two disjoint sets by $i$.
\end{proof}

\begin{lemma}
	$i \notin \llbracket i, 0 \rrparenthesis \wedge [i, i)$
\end{lemma}

\begin{proof}
	Both ranges define an empty range, and have no elements in the set of iterators therefore
	no element can be a member of an empty set.
\end{proof}

\begin{lemma}
	Empty ranges have neither first nor last elements.
\end{lemma}

\begin{proof}
	By Lemma 6.5 empty ranges have no elements, and therefore have no first nor last element.
\end{proof}

\begin{lemma}
	The size of a half-open weak range $\llbracket f, n \rrparenthesis$ is $n$. The size of a
	closed weak range $\llbracket f, n \rrbracket$ is $n + 1$. The size of a half-open bounded
	range $[f, l)$ is $l - f$. The size of a closed bounded range $[f, l]$ is $(l - f) + 1$.
\end{lemma}

\begin{exercise}
	Use \verb|find_if| and \verb|find_if_not| to implement quantifier functions \verb|all|,
	\verb|none|, \verb|not_all|, and \verb|some|, each taking a bounded range and a predicate.
\end{exercise}

\lstinputlisting[language=c++, firstline=1491, lastline=1525]{eop-code/eop.h}

\begin{exercise}
	Implement \verb|count_if| by passing an appropriate function object to \verb|for_each|
	and extracting the accumulation result from the returned function object.
\end{exercise}

\lstinputlisting[language=c++, firstline=1544, lastline=1562]{eop-code/eop.h}

\begin{exercise}
	Implement variations taking a weak range instead of a bounded range of all the versions
	of \verb|find|, quantifiers, \verb|count|, and \verb|reduce|.
\end{exercise}

\lstinputlisting[language=c++, firstline=1782, lastline=2083]{eop-code/eop.h}

\begin{exercise}
	State the postcondition for \verb|find_mismatch|, and explain why the final values
	of both iterators are returned.
\end{exercise}

\begin{solution}
	The postcondition for \verb|find_mismatch| is
	\[ f0 = l0 \vee f1 = l1 \vee \lnot r(\func{source}(f0), \func{source}(f1). \]
	Both iterators are returned since it allows the client to find the next mismatch.
	Another reason for returning both iterators is that it might not be possible
	to get back to the values returned since they we only require the \verb|Iterator|
	concept.
\end{solution}
