\chapter{Iterators}

\begin{lemma}
	$0 \leq j \leq i \wedge \func{weak\_range}(f, i) \Rightarrow \func{weak\_range}(f, j)$
\end{lemma}

\begin{proof}
	$(0 \leq k \leq j) \Rightarrow \func{successor}^k(f) \text{ is defined.}$
\end{proof}

\begin{lemma}
	$(f + n) + m = f + (n + m)$
\end{lemma}

\begin{proof}
	\begin{align*}
		f + 0 &= f\\
		f + \func{successor}(n) &= \func{successor}(\func{successor}^n(f))
	\end{align*}
\end{proof}

\begin{lemma}
	\verb|successor| is defined for every iterator in a half-open range and for every iterator
	except the last in a closed range.
\end{lemma}

\begin{lemma}
	If $i \in [f, l)$, both $[f, i)$ and $[i, l)$ are bounded ranges.
\end{lemma}

\begin{proof}
	The range $[f, l)$ is partitioned into two disjoint sets by $i$.
\end{proof}

\begin{lemma}
	$i \notin \llbracket i, 0 \rrparenthesis \wedge [i, i)$
\end{lemma}

\begin{proof}
	Both ranges define an empty range, and have no elements in the set of iterators therefore
	no element can be a member of an empty set.
\end{proof}

\begin{lemma}
	Empty ranges have neither first nor last elements.
\end{lemma}
